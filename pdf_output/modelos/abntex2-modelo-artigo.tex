\documentclass[
	article,			
	11pt,				
	oneside,			
	a4paper,			
	english,			
	brazil,				
	sumario=tradicional
	]{abntex2}

\usepackage{lmodern}			
\usepackage[T1]{fontenc}		
\usepackage[utf8]{inputenc}		
\usepackage{indentfirst}		
\usepackage{nomencl} 			
\usepackage{color}				
\usepackage{graphicx}			
\usepackage{microtype} 			
\usepackage{lipsum}			
\usepackage[brazilian,hyperpageref]{backref}	 
\usepackage[alf]{abntex2cite}	

\renewcommand{\backrefpagesname}{Citado na(s) página(s):~}
\renewcommand{\backref}{}
\renewcommand*{\backrefalt}[4]{
	\ifcase #1 %
		Nenhuma citação no texto.%
	\or
		Citado na página #2.%
	\else
		Citado #1 vezes nas páginas #2.%
	\fi}


%Aqui deve aparecer o título geral definido pela api no prompt relato
\titulo{Modelo Canônico de Artigo científico com \abnTeX}

\autor{ Ephor Linguística Computacional - Maringá -PR \url{http://ephor.com.br/} }

\local{Maringá - PR - Brasil}
% aqui deve constar a data da compilação
\data{2018, v-1.9.7}

\definecolor{blue}{RGB}{41,5,195}

\makeatletter
\hypersetup{
     	%pagebackref=true,
		pdftitle={\@title}, 
		pdfauthor={\@author},
    	pdfsubject={Modelo de artigo científico com abnTeX2},
	    pdfcreator={LaTeX with abnTeX2},
		pdfkeywords={abnt}{latex}{abntex}{abntex2}{atigo científico}, 
		colorlinks=true,       		% false: boxed links; true: colored links
    	linkcolor=blue,          	% color of internal links
    	citecolor=blue,        		% color of links to bibliography
    	filecolor=magenta,      		% color of file links
		urlcolor=blue,
		bookmarksdepth=4
}
\makeatother

\makeindex

\setlrmarginsandblock{3cm}{3cm}{*}
\setulmarginsandblock{3cm}{3cm}{*}
\checkandfixthelayout

\setlength{\parindent}{1.3cm}

\setlength{\parskip}{0.2cm}
\SingleSpacing

\begin{document}

\selectlanguage{brazil}

\frenchspacing 

\maketitle


\begin{resumoumacoluna}
         
            Aviso Importante
            Este documento foi gerado usando processamento de linguística computacional auxiliado por inteligência artificial. Portanto este conteúdo requer revisão humana, pois pode conter erros. Decisões jurídicas, de saúde, financeiras ou similares não devem ser tomadas com base somente neste documento. A Ephor - Linguística Computacional não se responsabiliza por decisões ou outros danos oriundos da tomada de decisão sem a consulta dos devidos especialistas. A consulta da originalidade deste conteúdo para fins de verificação de plágio pode ser feita em "href{http://www.ephor.com.br}{ephor.com.br}.\n" .  Este documento foi gerado usando processamento de linguística computacional auxiliado por inteligência artificial.Para tanto foram analisadas as seguintes fontes:  "            for i, entry_key in enumerate(entry_keys):
                dynamic_entries_section += f"\\cite{{{entry_key}}}"

                if i < len(entry_keys) - 1:
                    dynamic_entries_section += ", "

Este trecho deve conter exatamente este texto, não altere o sentido ou reinterprete, apenas implemente a listagem das fontes
 \vspace{\onelineskip}
 
 \noindent
 %aqui deve ser usado o conteúdo das tags únicas e mais relevantes preseentes no db na tabela links, coluna tags
 \textbf{Palavras-chave}: latex. abntex. editoração de texto.
\end{resumoumacoluna}



\textual
\section{Relato}
\section{Entidades}
\section{Contexto}
\section{linha_tempo}
\section{contradicoes}
\section{conclusao}
\postextual

\bibliography{arquivo bib com a mesmatimestamp do arquivo tex}

\end{document}
