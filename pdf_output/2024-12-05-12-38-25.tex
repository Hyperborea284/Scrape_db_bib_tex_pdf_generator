\documentclass[article,11pt,oneside,a4paper,brazil,sumario=tradicional]{abntex2}%
\usepackage[T1]{fontenc}%
\usepackage[utf8]{inputenc}%
\usepackage{lmodern}%
\usepackage{textcomp}%
\usepackage{lastpage}%
%
\usepackage[T1]{fontenc}%
\usepackage[utf8]{inputenc}%
\usepackage{lmodern}%
\usepackage{indentfirst}%
\usepackage{graphicx}%
\usepackage{color}%
\usepackage{microtype}%
\usepackage[brazilian,hyperpageref]{backref}%
\usepackage[alf]{abntex2cite}%

            \definecolor{blue}{RGB}{41,5,195}
            \hypersetup{
                pdftitle={Relatório Gerado},
                pdfauthor={Sistema de Gerenciamento},
                pdfsubject={Relatório gerado automaticamente},
                pdfkeywords={abnt, latex, abntex2, artigo científico},
                colorlinks=true,
                linkcolor=blue,
                citecolor=blue,
                urlcolor=blue
            }
            %
\newcommand{\theforeigntitle}{Generic Subtitle in a Foreign Language}%
\title{Relatório Gerado}%
\author{Sistema de Gerenciamento}%
\date{2024{-}12{-}05}%
\setlength{\parindent}{1.3cm}%
\setlength{\parskip}{0.2cm}%
\SingleSpacing%
%
\begin{document}%
\normalsize%
\maketitle%
\selectlanguage{brazil}%
\frenchspacing%
\section{Relato}%
\label{sec:Relato}%
Ap\'os o an\'uncio do presidente da Coreia do Sul, Yoon Suk Yeol, de decretar lei marcial, o Parlamento se reuniu e, de forma un\^anime, votou para rejeitar a medida. A revoga\c{c}\~ao da lei marcial foi seguida pela retirada das tropas do edif{\'\i}cio do Parlamento. Cidad\~aos que aguardavam do lado de fora do pr\'edio aplaudiram a not{\'\i}cia da vota\c{c}\~ao, expressando sua oposi\c{c}\~ao \`a medida e incentivando a desist\^encia do presidente. Yoon Suk Yeol justificou a declara\c{c}\~ao da lei marcial como uma medida para proteger a Coreia do Sul das ''for\c{c}as comunistas'' e eliminar elementos antiestatais. A medida inesperada ocorreu durante discuss\~oes sobre o projeto de lei or\c{c}ament\'aria do pr\'oximo ano entre o partido do presidente e o principal partido de oposi\c{c}\~ao.

%
\section{Contexto}%
\label{sec:Contexto}%
O presidente da Coreia do Sul, Yoon Suk Yeol, decretou a lei marcial para conter movimentos ''pr\'o-Coreia do Norte'', suspendendo direitos civis e limitando a atua\c{c}\~ao da imprensa. No entanto, sua decis\~ao foi revogada ap\'os uma vota\c{c}\~ao un\^anime do Parlamento sul-coreano. A revoga\c{c}\~ao da lei provocou rea\c{c}\~oes positivas entre a popula\c{c}\~ao, com soldados iniciando a retirada do edif{\'\i}cio do Parlamento. Yoon Suk Yeol justificou sua decis\~ao como um esfor\c{c}o para proteger a Coreia do Sul das ''for\c{c}as comunistas'' da Coreia do Norte. A declara\c{c}\~ao ocorreu em meio a tens\~oes pol{\'\i}ticas sobre o projeto de lei or\c{c}ament\'aria do pr\'oximo ano.

%
\section{Entidades}%
\label{sec:Entidades}%
1. O presidente da Coreia do Sul, Yoon Suk-yeol, decretou lei marcial no pa{\'\i}s, mas recuou ap\'os uma vota\c{c}\~ao un\^anime dos legisladores para bloquear a medida. A vota\c{c}\~ao foi seguida pela retirada das tropas que haviam entrado no edif{\'\i}cio do Parlamento, com os cidad\~aos comemorando a decis\~ao.

2. Yoon Suk-yeol justificou a declara\c{c}\~ao da lei marcial como uma medida para proteger a Coreia do Sul das ''for\c{c}as comunistas'' da Coreia do Norte e eliminar elementos antiestatais. Ele tamb\'em criticou a oposi\c{c}\~ao por, segundo ele, paralisar o governo devido a quest\~oes pol{\'\i}ticas e para proteger seu l{\'\i}der da justi\c{c}a.

3. A declara\c{c}\~ao de lei marcial ocorreu em meio \`as discuss\~oes entre o partido de Yoon e o principal partido de oposi\c{c}\~ao sobre o projeto de lei or\c{c}ament\'aria do pr\'oximo ano.

%
\section{Linha\_tempo}%
\label{sec:Linhatempo}%
Resumo para a se\c{c}\~ao linha do tempo:

- O presidente da Coreia do Sul, Yoon Suk Yeol, decretou lei marcial em resposta a amea\c{c}as consideradas como ''for\c{c}as comunistas'' da Coreia do Norte.
- Isso gerou rea\c{c}\~oes negativas, incluindo uma ''greve geral por tempo indeterminado'' liderada pelo maior sindicato do pa{\'\i}s, a Confedera\c{c}\~ao Coreana de Sindicatos (KCTU).
- No entanto, o presidente recuou da decis\~ao e suspendeu a lei marcial depois que o Parlamento votou de forma un\^anime para bloque\'a-la.
- A revoga\c{c}\~ao da lei marcial foi seguida pela sa{\'\i}da das tropas do edif{\'\i}cio do Parlamento, e a medida foi vista como controversa, enquanto surgiam memes nas redes sociais em resposta ao ocorrido.

%
\section{Contradicoes}%
\label{sec:Contradicoes}%
Desculpe, mas n\~ao h\'a texto especificado ao qual voc\^e gostaria de criar um resumo para a se\c{c}\~ao de contradi\c{c}\~oes. No entanto, se precisar de ajuda para resumir um texto espec{\'\i}fico, por favor, forne\c{c}a o texto e eu ficarei feliz em ajudar.

%
\section{Conclusao}%
\label{sec:Conclusao}%
A decis\~ao do presidente sul-coreano Yoon Suk Yeol de decretar a lei marcial foi revogada ap\'os uma vota\c{c}\~ao un\^anime pelo Parlamento sul-coreano. As tropas que haviam entrado no edif{\'\i}cio principal do Parlamento come\c{c}aram a se retirar ap\'os a vota\c{c}\~ao, e as esta\c{c}\~oes de televis\~ao locais mostraram tropas deixando o edif{\'\i}cio. O presidente havia decretado a lei marcial com o objetivo de proteger o pa{\'\i}s das ''for\c{c}as comunistas'' da Coreia do Norte. No entanto, a decis\~ao foi amplamente contestada, e a revoga\c{c}\~ao da lei marcial foi recebida com aplausos e comemora\c{c}\~ao por parte dos cidad\~aos presentes. Al\'em disso, o presidente expressou sua frustra\c{c}\~ao com a oposi\c{c}\~ao, acusando-os de paralisar o governo por motivos pol{\'\i}ticos. A medida ocorreu durante discuss\~oes sobre o projeto de lei or\c{c}ament\'aria do pr\'oximo ano, destacando a tens\~ao pol{\'\i}tica existente no pa{\'\i}s.

%
\bibliography{2024-12-05-12-38-25}%
\end{document}